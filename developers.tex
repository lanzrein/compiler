\documentclass{article}
\usepackage[utf8]{inputenc}
\usepackage{url}
\usepackage{listings}
\usepackage{hyperref}


\title{COM S 440 Project}
\author{Johan Lanzrein}
\date{Spring 2018}

\begin{document}

\maketitle

\section{Part 0 : Information about developer}

Name : Johan Lanzrein \\
URL for SSH : \url{git@git.linux.iastate.edu:lanzrein/coms440.git} \\
URL for HTTPS : \url{https://git.linux.iastate.edu/lanzrein/coms440.git}



\section{Part 1 : Lexer}
\subsection{Basic design}
We use the flex library to help us proceed the parsing. The repository for the library can be found here : \url{https://github.com/westes/flex}. The lexer is build using Makefile. Meaning a simple make builds the executable lexer that can be used as is. You need to have the flex library installed to be able to build the project. 
When launching the executable you can specify an argument that will be the file to be read from :  \begin{lstlisting}[language=bash]
./lexer <file> [debug]
\end{lstlisting}
Where file is the name of the file to read from. If no argument is given it will read from the standard input. You can also add the keyword debug at the end to have a more extensive report of how the file is read. This is mainly used when trying to find bugs. \\ \\
The flex library makes most the work when it comes to handling input files. Hence we only have to work on the how to process every token. \\ Therefore all of the parsing and processing is done in the $lexer.l$ file which is divided in four parts : Top, Definition, Rules and User Code.
\\subsubsection{Top}
This part contains all of the included library and files. 
It also contains a few global variables to help handling the preprocessing of \#include , \#define and \#ifdef. 
Finally it has a few prototype for helper functions. 
\subsubsection{Definition}
In here we define all of our regular expressions. Each one of them represents a token type. For example : 
\begin{lstlisting}[language=C]
IDENT [_a-zA-Z][_a-zA-Z0-9]*
\end{lstlisting}
 
We also define several states that will be of use for when we are including a file. The states are created like this : 
\begin{lstlisting}[language=C]
\%x INCLUDECND
\end{lstlisting}
\subsubsection{Rules}
This section contains all the rules on how we should act when we encounter every pattern defined in the previous section. Most of the patterns are just an instruction to print with a certain format on the standard output. 
Some other rules like the include, define, idef preprocessing instructions are longer. 
\begin{lstlisting}[language=C]
{LPAR}  {/*rule goes here */}
\end{lstlisting}
\subsubsection{User code}
This final part contains the main method from where the code is executed. There is also a helper function to help setting up include instructions. 
This is standard C code. 


\subsection{Files and description}
\label{definition}
\begin{description}
\item[lexer.l]
This is the file that will be processed by the flex library. We already described it in the previous section. However, we will insist on how it works when encountering preprocessing instructions in the next section. 
\item[tokens.c/tokens.h]
This file contains helper functions to print in a neat way what the lexer finds. It also has a few data structures that represents tokens. 
\item[defines.c/defines.h]
A helper file to take care of the defines preprocessing instructions. It has a data structure and helper methods.
\item[Makefile]
The Makefile helps to build the project with a single command. Nothing very special there. We also have a cleaning option available if you want to remove the intermediate files type "Make clean". 
\item[readme.md]
A readme that describes the files and how to build the lexer. 
\end{description}
\subsection{Data structures}
\subsubsection{Lexer.l}
\label{sec:lexerData}
\begin{lstlisting}[language=C]
 /**
 *@brief this structure contains data about a yybuffer
 *@param buffer the buffer
 *@param filename the filename of the buffer
 *@param lineno current line number
 */
typedef struct{
	YY_BUFFER_STATE buffer;
	char* filename;
	int lineno;
}buffer_data;
//A stack for our buffers to handle includes.
buffer_data buffer_stack[MAX_INCLUDE_DEPTH];
//Pointer to the current top of the stack 
int currPtr = 0;

\end{lstlisting}

For defines and if preprocessing, we also have data structures but they are both stacks with different names just like the buffer\_stack. Hence they are not shown here. 
\subsubsection{Tokens.h}
\begin{lstlisting}[language=C]
/**
 *@brief a structure to represent a token with its id and content
 */
struct token{
	const char* id;
	const char* content;
};
/*
 *@brief the id of the tokens
 */
const char * const TOKENS_MESSAGES[NUM_TOKENS];

\end{lstlisting}
\subsubsection{defines.h}
\label{sec:defineData}
\begin{lstlisting}[language=C]
 /**
  * @struct a basic structure to hold our defines. 
  * @param identifier the identifier of the define
  * @param arbitraryText of the identifier
  * */
 typedef struct {
	 char* identifier;
	 char* arbitraryText;
 }define;
 
  /**
   * @brief a data structure to help simulate a VLA ( like ArrayList ) 
   * @param size how many defines are done. 
   * @param array the actual array of defines
   * */
  typedef struct {
	int size;
	define* array;  
  }define_array;

  
\end{lstlisting}

\subsection{Specific methods used}
\subsubsection{Preprocessing includes}
All of the preprocessing of includes is done in the lexer.l file. Our lexer handles only the cases when the include is in a specific format. All other formating will lead to an error. 
The format expected is as follows : 
\begin{lstlisting}[language=C]
 #include "file.c" //file.c is the file to be included. 
\end{lstlisting}
Our include preprocessing can handle nesting of up to 1024 recursive calls, and will automatically detect include cycles. \\
How the program works is that it will maintain a stack (described in \autoref{sec:lexerData}) and will automatically update it as it encounters new include instruction or end-of-files. To detect cycles, it will check for name collisions and if a collision is encountered the whole program is exited. \\
This corresponds to coding the bonus parts of the assignment concerning include. 
\subsubsection{Preprocessing defines}
We have already described the data structures used to help working with defines (\autoref{sec:defineData}). 
When encountering defines instructions, the appropriate define data structure will be created and added to the list. On the other side when there is an undef instruction, the array is scaned and the corresponding define is removed. 
The program will check to not have twice the same identifier for a define and also for cycles very minimally. The program also supports recursive calls of define up to a depth of maximum 1028. 
\subsubsection{Preprocessing if}
Preprocessing if instructions is straight forward. When encountering an ifdef, we check if it is or not defined, and if it is continue scanning until and else or endif is encountered. In the other case we activate an option that forces to ignore furter input until and else or endif is encountered. \\
If encountering and else we do the opposite of what was done before ( ignoring $\Rightarrow$ reading, reading $\Rightarrow$ ignoring). If an endif is encountered, we set all flags used to 0 and continue reading. \\
In case of nested ifs, there is a stack that can handle at what level we are and what instructions to ignore. 

\section{Part 2 : Parser}
\subsection{Design}
We only add one file during this stage of the project. It is the file parser.y. With the help of the bison library, we can with relative ease write a grammar for our subset of the C language and decide if a given input file has a correct syntax or not. We also make some minor changes to the lexer file and update it so it links perfectly with the parser file. 
To use the program, like before we still have the Makefile that takes care of compiling and linking. It might produce a few warning messages because of shift-reduce conflicts, but as explained further we have taken care of them. 
Once the program is compiled do : \begin{lstlisting}[language=bash]
./parser <file> [debug]
\end{lstlisting}
Where file is name of the file to parse. You can add an option to debug which might be useful to track parsing errors. \\
On successful run, the parser won't produce any outcome. On error, the parser will output the type of error, the line where it was encountered as well as the file. Moreover it will say on what terminal the syntax error appeared. 
\subsection{Grammar}
We need to write a grammar for our C language. The grammar is detailed in the parser.y file but we transcript it in a more friendly to read format in our appendix \ref{grammar}. All the conflicts are solved using disambiguating rules. For instance the precedence of operators, or the dangling else problem is solved using a few helpful functions from the bison library. This allows to get rid of all shift / reduce or reduce / reduce conflicts that could appear while compiling the project. 
Note that we do not have the exact same form in the file parser.y, as the one described in appendix \ref{grammar}. They are nonetheless identical.\\
\subsection{Structures}
No particular structures have been added during this part of the project as we simply decide if an input file follows a particular grammar. 
The only outside method is the main method, which loads the file and sets up the data structures that might be needed for the lexer. \\


\section{Part 3 : Type checking}
This part concerns mainly the verification of an input program and also creates a basic abstract syntax tree that will be used further in the creation of our compiler. 
\subsection{Design}
The typechecking is a task that requires a lot more of coding and files than the parser or lexer. We therefore added four new files and their corresponding header files to help us handle this part. 
The main of the work is again handled by the file parser.y. On a specific rule the parser calls a method from a newly added file typechecking.c. This new file makes the computation to check if the operation is legal or not and returns an integer if the operation worked or not. Moreover, if a new identifier is declared or a new function is declared it is passed to the typechecking.c file that will store it in a symbol table datastructure. 
\subsection{Files and description}
\label{definition}
\begin{description}
\item{node.h and node.c}\\ These files contain the datastructure representing a node and all method necessary to handle node management. For example adding a left or right node, clear a node or a whole tree even. 
\item{functions.h and functions.c} \\ These files has the methods and datastructures to help managing the functions. It has a datastructure for a list of function that we use a rudimentary symbol table. And a datastructure for functions. All methods to create functions, and manage this symbol tables are in here. 
\item{identifiers.h and identifiers.c} \\These files work similarly to functions.h/functions.c. There is a list that is used as a symbol table for all identifiers, and a datastructure that represents an identifier. Again all methods to help handle the list and the identifier are coded here. 
\item{typechecking.h and typechecking.c} \\ These files work as an interface between the parser and the previously mentionned files. It contains two lists of identifiers : one for global identifiers, one for global identifiers. The file also has a list of function that have been declared. We will explain in more detail how this file works as it is the main helper to achieve typecheking. 
\end{description}

\subsection{Data Structures}
\subsubsection{node.h}
\begin{lstlisting}[language=C]
/**
 * @struct a structure representing a node of an AST
 * @param lef the left node
 * @param right the right node
 * @param type the type of the node if applicable 
 * @param sizeAttribute the size of the attribute
 * @param attribute. the attribute
 * */
typedef struct node{
	struct node* left;
	struct node* right;
	enum types type; //type of data if applicable NA is used for Operators. 
	size_t sizeAttribute;
	void* attribute;
}node;


\end{lstlisting}
\subsubsection{functions.h}
\begin{lstlisting}[language=C]
/**
 * @struct function carries information about functions
 * @param name the name of the function
 * @param argc the argument count of the function >=0 
 * @param lineDecl the line where the function is first declared
 * @param filename the name of the file where the function is declared
 * @param arguments the arguments stored as identifier
 * @param returnType the return type of the function 
 * */
typedef struct{
	char* name;
	size_t argc;
	int lineDecl;
	char* filename;
	identifier* arguments;
	enum types returnType;
}function;
/**
 * @struct a list to hold function
 * @param size the size of the list
 * @param functions a pointer to the list of functions. 
 * */
typedef struct {
	size_t size;
	function* functions;
}func_list;


\end{lstlisting}

\subsubsection{identifiers.h}
\begin{lstlisting}[language=C]
/**
 * @struct identifier represents an identifier with a few attributes
 * @param type the type of the identifier
 * @param name the name of the identifier
 * @param lineDecl the line where the identifier was declared
 * @param filename the name of the file where it was declared
 * */
typedef struct {
	enum types type;
	char* name; 
	int lineDecl;
	char* filename;
}identifier;
/**
 * @struct id_list a list of identifier
 * @param size the current size of the list
 * @param ids the identifiers in the list
 * */
typedef struct{
	size_t size;
	identifier* ids;
}id_list;
\end{lstlisting}
\subsubsection{tokens.h}
We add this enum to help process the different types. 
\begin{lstlisting}[language=C]
/**
 * @enum an enum holding the different type in our C language
 * */
 enum types{ NA,CHAR, INT, FLOAT , CHARARR, INTARR, FLOATARR };
 /**
  * @struct holds the translation in string of the types
  * */
 const char* typeTranslation[7];	

\end{lstlisting}

\subsection{Specific methods used}
\subsubsection{Typecheking}
As mentioned above, most the work of typechecking is done in the file typechecking.c. 
We will explain how to use it in order to take the most advantage of it on every part.  
\subsubsection{Functions}
When encountering a function, we can either add it to the list of function if it is a function declaration. Or if it is called/defined, we can use the method enter\_function that will setup to read the body of a function. It will create initialize the list of local identifiers, and explicit that we are in the body of the function so further identifiers are added in the correct scope. 
When exiting a function, remember to use the exit\_function method in order to clean up and check if the return type was given was the one expected. 
\subsubsection{Identifiers}
When encountering an identifier, we add them to either the list identifier\_local or identifier\_global depending on the scope. The scope is handled by the file typechecking. Hence all addition of any identifier is simply done by calling the method add\_id\_typecheking(). This will take care of adding in the correct scope. If an identifiers with the same name has already been declared in the same scope. The identifier won't be added and an error message will be displayed informing of the error. \\

\subsubsection{Expressions}
When encountering an expression, the program creates a node that it will try to link with a left a right node if appropriate. This allows to create a basic abstract syntax tree that will be made better when time allows in order to make further assignments easier. 
Any expression has a specific type. The program verifies that all operations that are performed with identifiers, constants and function calls are done within the rules defined in the instructions. 
If any error is detected the program will discard the line where the expression is and the operation that has failed will be reported in the console for the user to fix. 

\subsubsection{Function overloading}
Function overloading has been implemented. Meaning two function can have the same name and return type as long as their arguments not all their arguments match on their type. Since we have not yet implement widening of types this is fairly simple to implement on top of the basic code. We just need to change the code that verifies if two functions are equal or not. 

%%%%%%%%%%%%%%%%%%%%%%%%%%%%%%%%%%%%%%%%%%%%%%%%%%%%%%%%%%%%%%%%%%%%%%%%%%%%%

























%% APPENDIX 
\appendix
\section{\\Grammar of our C language subset}
\label{grammar}
We propose here a possible grammar for the task we have to do. Most of the terminals are written as is but the ones not explicitly written represent the tokens from part 1 of the project. For example {\sc SEMI} represents the semi-columns encountered. Moreover the non-terminal symbol {\%empty} represents an empty symbol. 
\subsection*{Grammar}

Let $program$ be the start symbol. 
\begin{eqnarray}
  \mathit{program} & \rightarrow & \mathit{sentences} \quad ;
  \\
    \mathit{sentences} & \rightarrow & \mathit{sentence} \quad \mathit{sentences} \quad  \\
    & | & \quad \mathit{sentence} \quad ;
    \\
      \mathit{sentence} & \rightarrow & \mathit{variabledeclaration} \quad \\
      & |& \mathit{functionprototype} \quad \\
      & |& \mathit{functiondeclaration} \quad ;
     %% About variable declaration 
\\
  \mathit{variabledeclaration} & \rightarrow & \mathit{TYPE} \quad \mathit{listidentifiers} \quad \mathit{SEMI} \quad ; 
  \\
  \mathit{listidentifiers} & \rightarrow & \mathit{declaration} \quad \mathit{,} \quad \mathit{listidentifiers} \quad   \\
  &  |& \mathit{declaration} \quad ; \\
  \mathit{declaration} & \rightarrow & \mathit{IDENT} \quad \mathit{[} \quad  \mathit{INTCONST} \quad \mathit{]} \quad  \\
  & | & \mathit{IDENT} ;  \\
  %% About function prototypes 
  \mathit{functionprototype} & \rightarrow &  \mathit{functiondeclaration}  \quad \mathit{SEMI} \quad ;\\
  \mathit{functiondeclaration} & \rightarrow & \mathit{TYPE} \quad \mathit{IDENT} \quad  \mathit{(} \quad\mathit{)} \\
  & | & \mathit{TYPE}\quad\mathit{IDENT} \quad \mathit{(} \quad \mathit{formallistparameters} \quad \mathit{)}  \\ 
   \mathit{formallistparameters} & \rightarrow & \mathit{formalparamater} \quad \mathit{formallistparameters}  | \\
   & | & \mathit{formalparameter} \\
   \mathit{formalparameter} & \rightarrow & \mathit{TYPE} \quad \mathit{IDENT}  \\
   & | & \mathit{TYPE} \quad \mathit{IDENT} \quad \mathit{[} \quad \mathit{]} ; \\
   \mathit{functiondefinition} & \Rightarrow & \mathit{functiondeclaration} \mathit{\{}\quad \mathit{body}  \quad \mathit{\}}  \\
   & | & \mathit{functiondeclaration} \quad \mathit{\{} \quad \mathit{\}} ;  \\
   \mathit{body} & \Rightarrow & \mathit{declorstat} \quad \mathit{body} \\
   & | & \mathit{declorstat} ; \\
   \mathit{declorstat} & \Rightarrow & \mathit{variabledeclaration} \\
   & | & \mathit{statement}\\
   & | & \mathit{statementblock} \\
   \mathit{statementblock} & \Rightarrow & \mathit{[} \quad \mathit{liststatements} \quad \mathit{]}\\
   & | & \mathit{[} \quad \mathit{]}\\
   \mathit{liststatement} & \Rightarrow & \mathit{statement} \quad \mathit{liststatement} \\
   & | & \mathit{statement} ; \\
   %% About statements....
   \mathit{statement} & \Rightarrow & \mathit{keywords} \quad \mathit{SEMI}\\
   & | & \mathit{RETURN} \quad \mathit{expression} \quad \mathit{SEMI}\\
   & | & \mathit{expression} \quad \mathit{SEMI} \\
   & | & \mathit{ifstatement} \\
   & | & \mathit{forstatement} \\
   & | & \mathit{whilestatement} \\
   & | & \mathit{dowhilestatement} ;\\
   \mathit{keywords} & \Rightarrow & \mathit{BREAK} \\
   & | & \mathit{CONTINUE};\\
   %%more on specific statements
   \mathit{ifstatement} & \Rightarrow & \mathit{IF} \mathit{(} \mathit{expression} \mathit{)} \mathit{statementorblock} \quad \mathit{elseblock} ;\\
   & | & \mathit{IF} \mathit{(} \mathit{expression} \mathit{)} \quad \mathit{statementorblock} \\
   \mathit{elseblock} & \Rightarrow & \mathit{ELSE} \quad \mathit{statementorblock} ; \\
   \mathit{statementorblock} & \Rightarrow & \mathit{statement} \\
   &| & \mathit{statementblock};\\
   %% for statement 
   \mathit{forstatement} & \Rightarrow & \mathit{FOR} \mathit{(} \quad \mathit{insideFor}\quad\mathit{)} \mathit{statementorblock};\\
   \mathit{insideFor} & \Rightarrow&\mathit{optionexpr}\mathit{SEMI}\mathit{optionexpr}\mathit{SEMI}\mathit{optionexpr};\\
   \mathit{optionexpr} & \Rightarrow & \mathit{expression} \\
   &  | & \mathit{\%empty};\\
   %%while
   \mathit{whilepart} & \Rightarrow & \mathit{WHILE}\quad \mathit{(} \quad \mathit{expression}\quad\mathit{)} ; \\
   \mathit{whilestatement} & \Rightarrow & \mathit{whilepart} \quad \mathit{statementorblock} \quad \mathit{SEMI} ; \\
   \mathit{dowhilestatement} & \Rightarrow & \mathit{DO} \quad \mathit{statementorblock} \quad \mathit{whilepart} \quad \mathit{SEMI} ; \\
   %%lvalues
   \mathit{lvalue} & \Rightarrow & \mathit{optionbracket} ; \\
   \mathit{optionbracket} & \Rightarrow & \mathit{[}\quad\mathit{expression}\quad\mathit{]}\\
   &|& \mathit{\%empty};\\
   %%expressions.. almost done ... biggest chunk
   \mathit{expression} & \Rightarrow & \mathit{IDENT}\\
    & | & \mathit{\&} \quad \mathit{IDENT}\\
    & | & \mathit{IDENT} \mathit{[}\quad\mathit{expression}\mathit{]}\\
    & | & \mathit{IDENT}\quad\mathit{(}\quad\mathit{)} \\
    & | & \mathit{IDENT} \quad \mathit{(} \quad\mathit{expressionlist} \mathit{)}\\
    & | & \mathit{lvalue} \quad \mathit{incrdecr} \\
    & | & \mathit{incredecr} \quad \mathit{lvalue} \\
    & | & \mathit{lvalue} \quad \mathit{=} \quad \mathit{expression}\\
    & | & \mathit{unaryop} \quad \mathit{expression}\\
    & | & \mathit{expression} \quad \mathit{binaryop} \quad \mathit{expression}\\
    & | & \mathit{expression} \mathit{?} \mathit{expression}\mathit{:}\mathit{expression}\\
    & | & \mathit{(}\quad\mathit{TYPE}\mathit{)}\quad\mathit{expression}\\
    & | & \mathit{(}\quad\mathit{expression}\quad\mathit{)}\\
    & | & \mathit{constant} ; \\
    \mathit{expressionlist} & \Rightarrow & \mathit{expression}\quad \mathit{,}\quad\mathit{expressionlist}\\
    & | & \mathit{expression};\\
    \mathit{incrdecr} & \Rightarrow & \mathit{++} \\
    & | & \mathit{--}\\
    \mathit{unaryop} & \Rightarrow & \mathit{-} \\
    & | & \mathit{!}\\
    & | & \mathit{\textasciitilde}\\
    \mathit{binaryop} & \Rightarrow & \mathit{==} \\
    & | & \mathit{!=}\\
    & | & \mathit{<} \\
    & | & \mathit{<=}\\
    & | & \mathit{>}\\
    & | & \mathit{>=}\\
    & | & \mathit{+}\\
    & | & \mathit{-}\\
    & | & \mathit{*}\\
    & | & \mathit{/}\\
    & | & \mathit{|}\\
    & | & \mathit{\&}\\
    & | & \mathit{\&\&}\\
    & | & \mathit{||};\\
    \mathit{constant} & \Rightarrow & \mathit{INTCONST}\\
    & | &\mathit{REALCONST}\\
    & | &\mathit{STRCONST}\\
    & | &\mathit{CHARCONST};
\end{eqnarray}
\end{document}
